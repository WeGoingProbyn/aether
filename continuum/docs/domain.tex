\documentclass[11pt]{article}

\usepackage[a4paper,margin=1in]{geometry}
\usepackage{amsmath,amssymb}
\usepackage{hyperref}

\title{Continuum: Grid, Topology, and Geometry}
\author{}
\date{}

\begin{document}
\maketitle

\section{Purpose and High-Level View}
This document describes the core mathematical roles of three systems in the \texttt{continuum} project:
\begin{itemize}
  \item \textbf{Grid}: represents the \emph{computational domain} (logical index space) and stores discrete fields.
  \item \textbf{Topology}: represents \emph{connectivity} between discrete entities (neighbours, boundaries, patch seams).
  \item \textbf{Geometry}: represents the \emph{mapping} between computational and physical space and provides metric quantities (volumes, face area-vectors) required by the solver.
\end{itemize}

The central design goal is to keep the finite-volume solver largely independent of the physical coordinate system or embedding: if the solver can query \emph{neighbour relations} (Topology) and \emph{geometric measures} (Geometry), it can advance conservative PDEs on many kinds of domains (Cartesian, curvilinear, multi-block, cubed-sphere, etc.).

\section{Computational vs Physical Domain}
Let $\boldsymbol{\xi}$ denote \emph{computational coordinates} (logical coordinates), e.g.
\[
\boldsymbol{\xi} = (\xi,\eta,\zeta) \in \hat{\Omega},
\]
where $\hat{\Omega}$ is typically a simple structured domain such as $[0,1]^d$. The \textbf{Grid} discretises $\hat{\Omega}$ with a structured index space $(i,j,k)$.

Let $\mathbf{x}$ denote \emph{physical coordinates} in the real domain $\Omega \subset \mathbb{R}^d$:
\[
\mathbf{x} = (x,y,z) \in \Omega.
\]
The \textbf{Geometry} provides a map
\[
\mathbf{x} = \mathbf{x}(\boldsymbol{\xi}),
\]
so that each logical cell in $\hat{\Omega}$ corresponds to a (possibly curved or distorted) physical control volume in $\Omega$.

\section{Finite Volume Update: What the Solver Needs}
Consider a conservative PDE in physical space,
\[
\frac{\partial U}{\partial t} + \nabla \cdot \mathbf{F}(U) = S(U,\mathbf{x},t),
\]
where $U$ is a vector of conserved quantities, $\mathbf{F}$ is a flux, and $S$ is a source term.

Finite volume methods evolve \emph{cell averages} over physical control volumes $V_i$:
\[
\bar{U}_i(t) = \frac{1}{|V_i|}\int_{V_i} U(\mathbf{x},t)\, dV.
\]
A standard semi-discrete FV form is:
\[
\frac{d}{dt}\left(|V_i|\,\bar{U}_i\right)
\;+\;
\sum_{f \in \partial V_i}
\int_{A_f} \mathbf{F}(U)\cdot \mathbf{n}\, dA
\;=\;
\int_{V_i} S\, dV.
\]

In practice, the solver typically needs:
\begin{itemize}
  \item \textbf{Cell volume} $|V_i|$
  \item \textbf{Face area-vector} $\mathbf{S}_{f,i} := \mathbf{n}_{f,i}\,|A_f|$ (physical normal times physical face area, oriented outward from cell $i$)
  \item \textbf{Neighbour info} for each face: which cell or boundary is adjacent to the face
\end{itemize}
The update step can be written schematically as
\[
\bar{U}_i^{n+1}
=
\bar{U}_i^{n}
-
\frac{\Delta t}{|V_i|}
\sum_{f \in \partial V_i}
\widehat{\mathbf{F}}_{f}\cdot \mathbf{S}_{f,i}
\;+\;
\Delta t\,\bar{S}_i,
\]
where $\widehat{\mathbf{F}}_{f}$ is a numerical flux (e.g.\ from a Riemann solver), computed using left/right states provided by Topology and reconstruction.

\section{Grid: The Computational Domain}
\subsection*{Role}
The \textbf{Grid} is the discrete representation of the computational domain $\hat{\Omega}$:
\begin{itemize}
  \item Defines the structured index space (2D or 3D), e.g.\ $(i,j,k)$ with extents $(N_x,N_y,N_z)$.
  \item Owns storage for discrete fields (cell-centered, face-centered, etc.).
  \item Provides iteration order and memory layout (AoS/SoA), ghost layers, and views/slices.
\end{itemize}

\subsection*{Key Idea}
A cell index $(i,j,k)$ is \emph{not} automatically a physical cube in $\Omega$. It is a \emph{logical} cell in $\hat{\Omega}$; its physical shape and size are provided by Geometry.

\section{Topology: Connectivity and Boundaries}
\subsection*{Role}
The \textbf{Topology} provides the relationships needed to traverse the grid as a control-volume complex:
\begin{itemize}
  \item For each cell, enumerates its faces and adjacent entities.
  \item For each face, identifies the neighbour cell across the face, or a boundary condition object (wall, inflow, outflow, periodic, etc.).
  \item For multi-block grids (e.g.\ cubed-sphere), encodes cross-block connections and index mappings at seams.
\end{itemize}

\subsection*{Key Idea}
Topology is \emph{purely combinatorial}: it answers ``who is adjacent to whom?'' without needing to know physical distances or areas. This lets the solver walk the mesh consistently, including at boundaries and patch interfaces.

\section{Geometry: Mapping and Metric Quantities}
\subsection*{Role}
The \textbf{Geometry} supplies physical meaning to logical grid entities by exposing metric quantities derived from the mapping $\mathbf{x}(\boldsymbol{\xi})$:
\begin{itemize}
  \item Computes/caches physical \textbf{cell volumes} $|V_i|$.
  \item Computes/caches physical \textbf{face area-vectors} $\mathbf{S}_{f,i}$.
  \item Optionally provides physical cell/face centers $\mathbf{x}_i$, $\mathbf{x}_f$ for source terms and reconstruction.
\end{itemize}

\subsection*{Jacobian Viewpoint}
Let $J$ be the Jacobian of the mapping:
\[
J = \frac{\partial \mathbf{x}}{\partial \boldsymbol{\xi}}.
\]
Then physical volume elements relate via
\[
dV = \left|\det J\right|\, d\boldsymbol{\xi}.
\]
In structured mapped grids, Geometry uses $J$ (and related cofactor data) to build the FV measures $|V_i|$ and $\mathbf{S}_{f,i}$ that appear in the flux balance.

\subsection*{Key Idea}
The solver should not need to know whether it is on Cartesian space, polar coordinates, or a cubed-sphere patch. If Geometry returns correct $(|V_i|, \mathbf{S}_{f,i})$, the same conservation update applies.

\section{Putting It Together}
At a high level, a solver timestep looks like:
\begin{enumerate}
  \item \textbf{Grid}: access current fields $\bar{U}_i^n$.
  \item \textbf{Topology}: for each cell $i$, iterate its faces and obtain neighbour/boundary information.
  \item \textbf{Geometry}: for each face and cell, query $\mathbf{S}_{f,i}$ and $|V_i|$ (and optionally centers).
  \item \textbf{Solver}: reconstruct left/right states, compute numerical flux $\widehat{\mathbf{F}}_f$, accumulate $\widehat{\mathbf{F}}_f\cdot \mathbf{S}_{f,i}$, divide by $|V_i|$, update $\bar{U}_i$.
\end{enumerate}

This separation supports extending \texttt{continuum} from simple Cartesian CFD to mapped domains (curvilinear grids, multi-block cubed-sphere) and, with richer Geometry, to more general coordinate systems where metric factors are required.

\end{document}
